% This file was converted to LaTeX by Writer2LaTeX ver. 1.9.9
% see http://writer2latex.sourceforge.net for more info
\documentclass{article}
\usepackage{calc,amsmath,amssymb,amsfonts}
\usepackage[LGR,T1]{fontenc}
\usepackage[greek,english]{babel}
\usepackage[style=numeric,backend=biber]{biblatex}
\date{2023-12-19}
\begin{document}

\bigskip


\bigskip


\bigskip


\bigskip


\bigskip


\bigskip


\bigskip

THE HISTORY OF FREE AND OPEN-SOURCE SOFTWARE


\bigskip


\bigskip


\bigskip


\bigskip


\bigskip


\bigskip


\bigskip

Trevor Arcieri

Computer Science 300: The Computing Professional

November 22, 2023


\bigskip


\bigskip


\bigskip


\bigskip


\bigskip


\bigskip

\clearpage

\ \ Open source software is a behemoth in today$\text{\textgreek{’}}$s software world, seen in the likes of Linux,
Python, Git, ChromeOS, and Firefox. Even today$\text{\textgreek{’}}$s most widely used operating system,
Android,\footnote{“Operating System Market Share Worldwide,” StatCounter Global Stats, 2023,
https://gs.statcounter.com/os-market-share\#monthly-202111-202303.} is open source, built on the Linux kernel. 

\ \ The premise sounds farcical: free software, for all to develop and use. How can free software available to all be
viable in a capitalist society? How can its integrity be maintained when anyone can be a developer? The answer is
simple: corporations stand to gain from continually developing the software since many use open-source liberally, and
the software$\text{\textgreek{’}}$s integrity is maintained by groups of volunteers which approve changes and create
releases, gaining experience and professional clout while having fun along the way. That leaves one question about open
source unanswered: how has open source risen to its current ubiquity?

\ \ Before addressing its history, it is crucial to define free and open-source software (FOSS). The Open Source
Initiative defines FOSS as software that: allows for free redistribution, includes the source code, allows
modifications and derived works to be distributed under the original license, may restrict source code distribution so
long as the license allows the distribution of “patch files” to modify the program at build time, and has a license
which does not discriminate against any person or group of persons, does not restrict program usage based off the field
in which it is used, applies to all to whom the program is redistributed without the need of an additional license,
does not depend on the program$\text{\textgreek{’}}$s being part of a particular software distribution, does not place
restrictions on other software distributed along with the licensed software, and is not predicated on any individual
technology.\footnote{“The Open Source Definition,” Open Source Initiative, February 22, 2023,
https://opensource.org/osd/.}

\ \ In short, FOSS can be described as software which may be acquired at no cost, with publicized source code, and
without limitations on derived works. This last aspect is key to the open source definition, because it ensures that
FOSS is always free, regardless of what software distribution it may be a part of, thus ensuring that all FOSS can be
used in other FOSS projects, as well as in proprietary software. This provides a commercially-viable quality to FOSS:
it can be used inside software which is not open source, which is the reason open-source lies at the core of so many
modern commercial software products.

\ \ So, how did FOSS begin? The history of sharing technologies tracks all the way back to the early 1900s in the
automotive industry. In 1879, a man named George B. Selden filed for a patent on the 2-stroke gasoline engine. In 1899,
he sold the patent to William C. Whitney, who created the Electric Vehicle Company (EVC). Selden and Whitney then began
collecting royalties from automobile manufacturers using the 2-stroke engine in their vehicles. In 1903, Henry Ford,
owner of Ford Motor Company, along with four other automobile manufacturers, contested a patent infringement suit from
Selden and EVC. The ensuing legal battle would last eight years, with Ford ultimately coming out victorious in
1911.\footnote{William Greenleaf and David L. Lewis, Monopoly on Wheels: Henry Ford and the Selden Automobile Patent
(Detroit, MI: Wayne State University Press, 2011).}

\ \ Following Ford$\text{\textgreek{’}}$s victory, an association which would come to be known as the Automobile
Manufacturers Association was formed.\footnote{Tom Mahoney, The Story of George Romney: Builder, Salesman, Crusader
(Whitefish, MT: Literary Licensing, 2012).} The association prescribed a cross-licensing agreement between
participating auto manufacturers which allowed for the sharing of technology. Each company would develop its own
technology and patents, allowing the other manufacturers to use them freely.\footnote{James J. Flink, The Car Culture
(Cambridge, MA: MIT Press, 1987).} This thesis of developing technology and sharing between corporations to advance the
industry via collaboration lies at the core of today$\text{\textgreek{’}}$s FOSS.

\ \ The concept of free software was inherent to software$\text{\textgreek{’}}$s beginnings in academia. In the 1950s
and 60s, software was largely used in cutting-edge areas of research. Just as most cutting-edge research topics, much
of the work was done in coordination between academic and corporate researchers.\footnote{Catharina Maracke, “Free and
Open Source Software and Frand‐based Patent Licenses,” The Journal of World Intellectual Property 22, no. 3–4 (2019):
78–102, https://doi.org/10.1111/jwip.12114.} Thus, the software was often published in the public domain without any
restrictions, making it shared even more liberally than modern FOSS. 

\ \ Additionally, most contemporary source code was distributed alongside its machine code counterpart, because users
would modify the source code to be used on different hardware or operating systems, as well as to fix issues and
increase functionality for their application.\footnote{Eric von Hippel and Georg von Krogh, “Open Source Software and
the $\text{\textgreek{‘}}$Private-Collective$\text{\textgreek{’}}$ Innovation Model: Issues for Organization Science,”
Organization Science 14, no. 2 (April 1, 2003): 209–23, https://doi.org/10.1287/orsc.14.2.209.14992.} An early form of
such open source software came to exist in the A-2 System developed at UNIVAC in 1953,\footnote{Paul E. Ceruzzi, A
History of Modern Computing, 1945-1995 (Cambridge, MA: MIT Press, 1998).} based on the A-0 System developed by Grace
Murray Hopper.\footnote{Richard K. Ridgway, “Compiling Routines,” Proceedings of the 1952 ACM National Meeting
(Toronto) on  - ACM $\text{\textgreek{’}}$52, 1952, 1–5, https://doi.org/10.1145/800259.808980.} The source code was
released alongside the A-2 System to customers who were invited to send their improvements back to
UNIVAC,\footnote{Douglas Crockford, “Heresy \& Heretical Open Source: A Heretic$\text{\textgreek{’}}$s Perspective,”
InfoQ, March 11, 2011, https://www.infoq.com/presentations/Heretical-Open-Source/.} encouraging community
collaboration. Similarly, nearly all software distributed for IBM mainframe computers included source code.

\ \ As software was distributed more widely, communities formed around the software to enhance collaborative development
efforts. One notable example of these early open-source communities is SHARE, which was founded in 1955 by Los-Angeles
users of the IBM 704 mainframe computer.\footnote{“Share (Computing),” Wikipedia, May 12, 2023,
https://en.wikipedia.org/wiki/SHARE\_(computing).} SHARE$\text{\textgreek{’}}$s SHARE library allowed for the exchange
of source code modifications between programmers, encouraging distributed development of the
modifications.\footnote{David Gardner, “SHARE, IBM User Group, To Celebrate 50th Anniversary,” TechWeb News, August 17,
2005.} In 1959, SHARE released its own SHARE Operating System for the IBM 709 mainframe computer, which was
collaboratively developed as an improvement on an operating system from General Motors for the IBM 704.\footnote{“Share
Operating System,” Wikipedia, August 13, 2023, https://en.wikipedia.org/wiki/SHARE\_Operating\_System.}

\ \ Another such early open source community was the Digital Equipment Computer Users$\text{\textgreek{’}}$s Society
(DECUS), a group of users of computers from the Digital Equipment Corporation,\footnote{“DECUS,” Wikipedia, June 10,
2023, https://en.wikipedia.org/wiki/DECUS.} a major player in the early computer industry.\footnote{“Digital Equipment
Corporation,” Wikipedia, November 19, 2023, https://en.wikipedia.org/wiki/Digital\_Equipment\_Corporation.} DECUS
assembled a library of software from program submitters and distributed to any
requesters,\footnote{“DECUSLIBrary.Compendium,” decuslib.com, 2023, http://www.decuslib.com/.} publishing catalogs of
the available software annually.\footnote{DECUS, “DECUS\_Catalog\_PDP-11\_Aug78” (DECUS, August 1978).}

\ \ A more generally-focused example of an open-source community was the Advanced Research Projects Agency Network
(ARPANET). In 1969, it was erected as the “first transcontinental, high-speed computer network,” and would later be
succeeded by the internet. The network linked together hundreds of universities, research laboratories, and
corporations, allowing developers to share source code widely.\footnote{Hippel and Krogh, “Open Source Software.”}

\ \ With this initial boom of free software, some issues began to appear. Operating systems and compilers were growing
more complex, and the cost of software development saw a marked increase relative to hardware development. As some
software began to be sold under restricted licenses, the industry faced fierce competition with software that came
bundled with hardware. 

\ \ However, this competition would be altered drastically. Leased machines required continuous software support without
any continuous revenue, which encouraged companies to cease providing software improvements for free. Some customers
that desired to modify software were against bundled software, since software products would allow more freedom in this
regard. Finally, in 1969 the antitrust suit United States v. IBM resulted in the US striking down bundled software as
anticompetitive.\footnote{Franklin M. Fisher, James W. Mackie, and Richard B. Mancke, essay, in IBM and the U.S. Data
Processing Industry: An Economic History (Praeger, 1983), 176.} Thus, the technology market was left with a need to
sell software and the market opening to do so, causing a stark increase in the amount of software sold as a product,
still under restricted licenses.

\ \ The software industry then faced a rather large hurdle: while software could be sold under a restricted license, it
was unable to be copyrighted. Thus, a program could belong to a company, be loaned out for a prescribed usage by
another company, but there was no intellectual property in a program. To address this issue, the US Congress released
the 1974 US Commission on New Technological Uses of Copyrighted Works (CONTU), which established that source code is a
proper subject of copyright. 

\ \ Following, in the 1983 case Apple v. Franklin, a US appellate court decided that software machine code was valid
under copyright as well. Ultimately, this provided computer programs the copyright status of literary works, enshrining
the integrity of the intellectual property and licensing of software in legal doctrine.\footnote{Jan L. Nussbaum,
“Apple Computer, Inc. v. Franklin Computer Corporation Puts the Byte Back into Copyright Protection for Computer
Programs,” Golden Gate University Law Review 14, no. 2 (1984): 281–308.} With the newfound legal fortitude of programs,
the paid software industry faced a massive boom, while free software saw a sharp decline.

\ \ Many companies began charging for software licenses and legally restricting software development. In 1979, AT\&T
began enforcing its licenses to prevent the free distribution of Unix and began selling licenses to use the software,
which was previously used at no cost by the government and in academia.\footnote{Steven Weber, essay, in The Success of
Open Source (Cambridge, MA: Harvard University Press, 2005), 38–44.} Similarly, in early 1983, IBM stopped distributing
source code along with its software products altogether.\footnote{DISTRIBUTION OF IBM LICENSED PROGRAMS AND LICENSED
PROGRAM MATERIALS AND MODIFIED AGREEMENT FOR IBM LICENSED PROGRAMS, February 8, 1983, IBM,
http://landley.net/history/mirror/ibm/oco.html.} This spurned a trend of corporations refrained from distributing
source code alongside their machine code to increase revenues.

\ \ As companies began to restrict source code, a myriad of methods for sharing source code among hobbyist programmers
came to fruition. In the early 1980s, DECUS began assembling entire distributions of submitted programs on magnetic
tapes, known as the DECUS tapes, which could be disseminated to DECUS members upon request.\footnote{“The DECUS Tapes,”
Index of /pub/academic/computer-science/history/PDP-11/RSX/decus, accessed November 22, 2023,
http://www.ibiblio.org/pub/academic/computer-science/history/pdp-11/rsx/decus/.} With computer networks growing in
popularity and usage in the early 1980s, virtual sharing systems began to flourish as well.

\ \ Bulletin Board System (BBS) networks were a prime example. These were servers that allowed users to connect via
terminal, log in, and exchange software and data on public message boards.\footnote{Frank Derfler, “Dial Up Directory,”
Kilobaud Microcomputing Magazine, April 1, 1980.} One of the most popular brands of BBS was WWIV, which distributed its
source code along with the BBS, allowing administrators to modify the BBS to their specific needs. WWIVnet was a
network for WWIV BBSes, allowing thousands to be linked to the same network at once, thus enhancing the potential for
virtual collaboration on software.\footnote{Wayne Bell, “The Official History of WWIV,” WWIVNEWS 1, no. 1 (January
1991).}

\ \ A more decentralized version of BBS, User$\text{\textgreek{’}}$s Network (USENET), was developed in
1980.\footnote{Erik Gregersen, “Usenet,” Encyclopædia Britannica, January 17, 2023,
https://www.britannica.com/technology/USENET.} Similarly, it allowed reading and posting on public message boards, but
posts were stored and forwarded between a changing set of servers. Unix-to-Unix Copy Protocol (UUCP) was used to
interconnect the servers and to connect them to users, creating a worldwide network of computers connected by UUCP,
known as UUCPnet.\footnote{“Usenet,” Wikipedia, November 18, 2023, https://en.wikipedia.org/wiki/Usenet.} UUCPnet thus
allowed for widespread, global software collaboration.

\ \ Armed with these collaborative development networks, Richard Stallman anounced the collaborative GNU Project in
1983. In his announcement, he stated his reasoning for the GNU project: he believed any non-free software to be
unethical, and wanted to develop Unix-compatible operating system made of entirely free software that he could use
without moral conflict.\footnote{Richard Stallman, “Initial Announcement - GNU Project - Free Software Foundation,”
Initial Announcement, September 27, 1983, https://www.gnu.org/gnu/initial-announcement.html.} This marked the beginning
of the first entirely free operating system and, arguably, one of the most influential open source projects in history.

\ \ Following his announcement of the GNU project, Stallman published the GNU Manifesto in 1985, outlining the purpose
of the GNU project in creating free software and requesting assistance in the project$\text{\textgreek{’}}$s
development.\footnote{Richard Stallman, “The GNU Manifesto - GNU Project - Free Software Foundation,” The GNU
Manifesto, March 1985, https://www.gnu.org/gnu/manifesto.en.html.} Later that year, Stallman established the Free
Software Foundation (FSF), dedicated to promoting free software.\footnote{William L Hosch, “Free Software Foundation,”
Encyclopædia Britannica, May 11, 2023, https://www.britannica.com/topic/Free-Software-Foundation.} In 1989, the FSF
released the GNU General Public License (GPL), which not only ensured that GNU and any works derived from it would stay
free in perpetuity, but also was made freely available to be used in other free software projects.\footnote{“GNU
General Public License,” The GNU Operating System and the Free Software Movement, February 1989,
https://www.gnu.org/licenses/old-licenses/gpl-1.0.txt.} This became known as copyleft, where a copyright license was
used to ensure the free distribution of the licensed software, rather than limit to limit its distribution as copyright
is intended for. Importantly, copyleft does not restrict entities from charging for the software, but ensures that
users have the freedom to distribute and modify the software.\footnote{“What Is Copyleft? - GNU Project - Free Software
Foundation,” What is Copyleft?, 2022, https://www.gnu.org/licenses/copyleft.html.}

\ \ The GNU kernel, GNU Hurd, was delayed significantly, and was still unimplemented when most other aspects of GNU had
been completed in 1991. Seeking a fully free operating system, Linus Torvalds released the first open-source kernel,
Linux, in early 1992 under the GNU GPL.\footnote{“RELEASE NOTES FOR LINUX v0.12,” Mirrors.kernel.org, 1992,
https://mirrors.edge.kernel.org/pub/linux/kernel/Historic/old-versions/RELNOTES-0.12.} Thus, the first completely free
operating system was born, and Linus Torvalds and Richard Stallman$\text{\textgreek{’}}$s names were cemented in
history. The GNU/Linux operating system played an integral role in proving the viability of collaborative software
development, as both GNU and Linux utilized the efforts of many volunteer developers. Crucially, as businesses began to
adopt develop on top of GNU/Linux in the 1990s and 2000s, the operating system proved the commercial integrity of FOSS.

\ \ Much of the GNU and Linux projects can be described as free software, since this is the name Stallman coined in 1985
with the Free Software Foundation. The term “open source” was not created until 1998, at a strategy session in Palo
Alto, California. It was held after an announcement of the release of the Netscape source code, the company behind the
most prominent contemporary web browser. The strategy session was held with the thought that the publicity surrounding
the Netscape announcement provided a unique opportunity to promote an open development
process.\footnote{Opensource.org, “History of the OSI,” Open Source Initiative, September 19, 2006,
https://opensource.org/history/.}

\ \ The conferees sought to create a single label that identified the approach of community-sourced code improvements
while expressing the commercial usability of such an approach, without the extreme ideology (that all software should
entirely be free) behind the Free Software Movement. Christine Peterson, in attendance, proposed the term “open
source.”\footnote{Opensource.org, “History of the OSI.”} This term accomplished the goal in its entirety: “open”
appears much more commercial-friendly than “free,” and “open” does not carry the ideological baggage of “free.”

\ \ Following the creation of “open source,” Linus Torvalds expressed his early support, greatly benefiting the
rebranding effort.\footnote{Opensource.org, “History of the OSI.”} Richard Stallman, on the other hand, was against the
term because it did not strictly advocate for the principle of free software, and remained in favor of the term “free
software” to make that meaning clear.\footnote{Richard Stallman, “Why Open Source Misses the Point of Free Software -
GNU Project - Free Software Foundation,” Why Open Source Misses the Point of Free Software, 2023,
https://www.gnu.org/philosophy/open-source-misses-the-point.en.html.} An event organized in 1998 by Time
O$\text{\textgreek{’}}$Rielly was slated to be called the “Freeware Summit,” but he renamed it to the “Open Source
Summit,”\footnote{Guido van Rossum, “Open Source Summit Trip Report,” Open Source Summit LG \#28, May 1998,
https://linuxgazette.net/issue28/rossum.html.} cementing the term “open source” in the popular mind.\footnote{B.
Fitzgerald and P.J. Agerfalk, “The Mysteries of Open Source Software: Black and White and Red All Over?,” Proceedings
of the 38th Annual Hawaii International Conference on System Sciences 8 (2005),
https://doi.org/10.1109/hicss.2005.609.} Soon after the Open Source Summit, The Open Source Initiative was founded. The
organization continues to educate and advocate for open source to this day.\footnote{Opensource.org, “History of the
OSI.”}

\ \ Now that open source had grown in popularity, a few large-scale FOSS projects began to spring up. One early project
of note was the X Window System, created in 1984. It used a networking protocol to provided a basic FOSS framework for
a multi-window computer GUI environment without mandating any particular UI inside the windows, using a keyboard and
mouse as the interface.\footnote{Robert W. Scheifler and Jim Gettys, “The X Window System,” ACM Transactions on
Graphics 5, no. 2 (1986): 79–109, https://doi.org/10.1145/22949.24053.} Another along this line was the GNU Network
Object Model Environment (GNOME). GNOME was developed in 1997 as a FOSS successor to a mostly open-source desktop
environment, KDE. GNOME is a full desktop environment with a free and complete set of applications that have a
standardized UI,\footnote{Miguel de Icaza, “The GNOME Desktop Project.,” The GNOME Desktop Project., August 15, 1997,
https://mail.gnome.org/archives/gtk-list/1997-August/msg00123.html.} and is still utilized in many major Linux
distributions today.

\ \ With substantial FOSS projects gaining traction, there was some early resistance to the movement. A series of legal
battles were initiated by Microsoft-funded Unix and Linux distribution vendor SCO. One of the largest battles among
these was SCO v. IBM, filed in 2003. In it, SCO alleged that in its contributions to the Linux kernel, IBM had copied
intellectual property from the Unix kernel.\footnote{SCO GROUP, INC. v. INTERN. BUSINESS MACHINES CORP.
(https://www.leagle.com/decision/infco20180102045 January 2, 2018).} The situation escalated quickly, with Red Hat and
IBM taking legal action against SCO, SCO threatening legal action against Linux users without a Unix license, and SCO
suing Novell who claimed they own the IP of the Unix kernel.

\ \ In 2006, the SCO v. IBM trial date was vacated pending the results of SCO v. Novell.\footnote{SCO v. IBM.} Then, in
2007, the court found that Novell, not SCO, owned the intellectual property rights over the Unix kernel,\footnote{John
Markoff, “Judge Says Unix Copyrights Rightfully Belong to Novell,” The New York Times, August 11, 2007,
https://www.nytimes.com/2007/08/11/technology/11novell.html.} and that SCO was “obligated to recognize Novell's waiver
of SCO's claims against IBM.”\footnote{“Court Rules: Novell Owns the Unix and Unixware Copyrights! Novell Has Right to
Waive!,” Groklaw, August 10, 2007, http://www.groklaw.net/article.php?story=20070810165237718.} \footnote{THE SCO
GROUP, INC. v. NOVELL, INC. SCO Grp v. Novell Inc (https://www.leagle.com/decision/infdco20110204657 June 10,
2010).}\ Further, following the court ruling, Novell stated they will not pursue legal action against Linux users,
saying “We don't believe there is Unix in Linux.”\footnote{Elizabeth Montalbano, “Novell Won$\text{\textgreek{’}}$t
Pursue Unix Copyrights,” PCWorld - Novell Won$\text{\textgreek{’}}$t Pursue Unix Copyrights, August 15, 2007,
http://www.pcworld.com/article/id\%2C135959-c\%2Cunix/article.html.} Consequently, SCO filed for bankruptcy and the SCO
v. IBM case was administratively closed. While the case did leave the legality of using the Linux kernel in tumult for
many years, at its resolution the legality of the Linux kernel was definitive, and that of FOSS was greatly bolstered
as a whole.

\ \ Microsoft itself has a controversial history with FOSS. In the early 2000s, Microsoft was vehemently against FOSS,
with its contemporary CEO, Steve Ballmer, publicly referring to Linux as a “malignant cancer.” In 2008, a week before
he would retire, Bill Gates insisted to the leadership at Microsoft that they adopt open source. By 2012, Microsoft
launched Microsoft Open Technologies Inc., with the goal of interfacing between Microsoft$\text{\textgreek{’}}$s
proprietary software and other FOSS.\footnote{Shira Ovide, “Microsoft Dips Further into Open-Source,” The Wall Street
Journal, April 17, 2012, https://www.wsj.com/articles/SB10001424052702304432704577347783238850756.} By 2014, Microsoft
CEO Satya Nadella even announced Microsoft$\text{\textgreek{’}}$s reconciliation and further commitment to Linux,
stating in a presentation “Microsoft loves Linux.”\footnote{Microsoft Windows Server Team, “Microsoft Loves Linux,”
Microsoft Windows Server Blog, September 3, 2021,
https://cloudblogs.microsoft.com/windowsserver/2015/05/06/microsoft-loves-linux/.}

\ \ This brings the story to today, a time when Windows Subsystem for Linux can be installed freely to run Linux on any
Windows operating system. The state of FOSS today is lively and ever-growing, with many developers contributing to FOSS
in their free time to increase their technical acumen and rapport. The active state of FOSS today would not be possible
without the intricate FOSS history it is built on. 

\ \ FOSS had a long journey from beginnings in the automotive industry, to academic collaboration, to more closed-source
software empowered by software copyrights. Even while closed-source was gaining popularity hobbyists continued sharing
their programs, with Richard Stallman creating the GNU project, allowing for a fully free operating system when
combined with Linus Torvalds$\text{\textgreek{’}}$s Linux. It was at this point that the term “open source” took off,
and open source projects such as the X Window System and GNOME popped up. Some corporations such as SCO and Microsoft
tried to push back against the movement, but all either relinquished or joined in. 

\ \ Ultimately, its history led FOSS to become the behemoth it is today: healthy and providing a backbone to much of the
software enjoyed by users the world over.

\clearpage
Bibliography

“A-0 System.” Wikipedia, August 23, 2023. https://en.wikipedia.org/wiki/A-0\_System. 

“Automobile Manufacturers Association.” Wikipedia, May 21, 2023. https://
en.wikipedia.org/wiki/Automobile\_Manufacturers\_Association. 

Bell, Wayne. “The Official History of WWIV.” WWIVNEWS 1, no. 1 (January 1991). 

“Bulletin Board System.” Wikipedia, November 11, 2023. https://en.wikipe dia.org/wiki/Bulletin\_board\_system. 

Ceruzzi, Paul E. A history of modern computing, 1945-1995. Cambridge, MA: MIT Press, 1998.

“Court Rules: Novell Owns the Unix and Unixware Copyrights! Novell Has Right to Waive!” Groklaw, August 10, 2007.
http://www.groklaw.net/article.ph p?story=20070810165237718. 

Crockford, Douglas. “Heresy \& Heretical Open Source: A Heretic$\text{\textgreek{’}}$s Perspective.” InfoQ, March 11,
2011. https://www.infoq.com/presentations/Heretical-Open-Source/. 

DECUS. “DECUS\_Catalog\_PDP-11\_Aug78.” DECUS, August 1978. 

“DECUS.” Wikipedia, June 10, 2023. https://en.wikipedia.org/wiki/DECUS. 

“DECUSLIBrary.Compendium.” decuslib.com, 2023. http://www.decuslib. com/. 

Derfler, Frank. “Dial Up Directory.” Kilobaud Microcomputing Magazine, April 1, 1980. 

“Digital Equipment Corporation.” Wikipedia, November 19, 2023. https://
en.wikipedia.org/wiki/Digital\_Equipment\_Corporation. 

DISTRIBUTION OF IBM LICENSED PROGRAMS AND LICENSED PROGRAM MATERIALS AND MODIFIED AGREEMENT FOR IBM LICENSED PROGRAMS,
February 8, 1983. IBM. http://landley.net/history/mirror/ibm/ oco.html. 

Fisher, Franklin M., James W. Mackie, and Richard B. Mancke. Essay. In IBM and the U.S. Data Processing Industry: An
Economic History, 176. Praeger, 1983. 

Fitzgerald, B., and P.J. Agerfalk. “The Mysteries of Open Source Software: Black and White and Red All Over?”
Proceedings of the 38th Annual Hawaii International Conference on System Sciences 8 (2005). https://doi.org/10.1109/
hicss.2005.609. 

Flink, James J. The car culture. Cambridge, MA: MIT Press, 1987. 

Gardner, David. “SHARE, IBM User Group, To Celebrate 50th Anniversary.” TechWeb News, August 17, 2005. 

“George B. Selden.” Wikipedia, March 23, 2023. https://en.wikipedia.org/ wiki/George\_B.\_Selden. 

“GNU General Public License.” The GNU Operating System and the Free Software Movement, February 1989.
https://www.gnu.org/licenses/old-licenses /gpl-1.0.txt. 

Greenleaf, William, and David L. Lewis. Monopoly on wheels: Henry Ford and the Selden Automobile Patent. Detroit, MI:
Wayne State University Press, 2011. 

Gregersen, Erik. “Usenet.” Encyclopædia Britannica, January 17, 2023. https://www.britannica.com/technology/USENET. 

Hippel, Eric von, and Georg von Krogh. “Open Source Software and the
$\text{\textgreek{‘}}$Private-Collective$\text{\textgreek{’}}$ Innovation Model: Issues for Organization Science.”
Organization Science 14, no. 2 (April 1, 2003): 209–23. https://doi.org/10.1287/orsc. 14.2.209.14992. 

“History of Free and Open-Source Software.” Wikipedia, June 28, 2023.
https://en.wikipedia.org/wiki/History\_of\_free\_and\_open-source\_software. 

Hosch, William L. “Free Software Foundation.” Encyclopædia Britannica, May 11, 2023.
https://www.britannica.com/topic/Free-Software-Foundation. 

Icaza, Miguel de. “The GNOME Desktop Project.” The GNOME Desktop Project., August 15, 1997.
https://mail.gnome.org/archives/gtk-list/1997-August/msg00123.html. 

Mahoney, Tom. The story of George Romney: Builder, salesman, Crusader. Whitefish, MT: Literary Licensing, 2012. 

Maracke, Catharina. “Free and Open Source Software and Frand‐based Patent Licenses.” The Journal of World Intellectual
Property 22, no. 3–4 (2019): 78–102. https://doi.org/10.1111/jwip.12114. 

Markoff, John. “Judge Says Unix Copyrights Rightfully Belong to Novell.” The New York Times, August 11, 2007.
https://www.nytimes.com/2007/08/11 /technology/11novell.html. 

Metz, Cade. “Meet Bill Gates, the Man Who Changed Open Source Software.” Wired, January 30, 2012.
https://www.wired.com/2012/01/meet-bill-gates/. 

Microsoft Windows Server Team. “Microsoft Loves Linux.” Microsoft Windows Server Blog, September 3, 2021.
https://cloudblogs.microsoft.com/window sserver/2015/05/06/microsoft-loves-linux/. 

Montalbano, Elizabeth. “Novell Won$\text{\textgreek{’}}$t Pursue Unix Copyrights.” PCWorld - Novell
Won$\text{\textgreek{’}}$t Pursue Unix Copyrights, August 15, 2007. http://www.pcworld.
com/article/id\%2C135959-c\%2Cunix/article.html. 

Nussbaum, Jan L. “Apple Computer, Inc. v. Franklin Computer Corporation Puts the Byte Back into Copyright Protection for
Computer Programs.” Golden Gate University Law Review 14, no. 2 (1984): 281–308. 

Opensource.org. “History of the OSI.” Open Source Initiative, September 19, 2006. https://opensource.org/history/. 

Opensource.org. “The Open Source Definition.” Open Source Initiative, February 22, 2023. https://opensource.org/osd/. 

“Operating System Market Share Worldwide.” StatCounter Global Stats, 2023.
https://gs.statcounter.com/os-market-share\#monthly-202111-202303. 

Ovide, Shira. “Microsoft Dips Further into Open-Source.” The Wall Street Journal, April 17, 2012.
https://www.wsj.com/articles/SB10001424052702304 432704577347783238850756. 

“RELEASE NOTES FOR LINUX v0.12.” Mirrors.kernel.org, 1992. https://
mirrors.edge.kernel.org/pub/linux/kernel/Historic/old-versions/RELNOTES-0. 12. 

Ridgway, Richard K. “Compiling Routines.” Proceedings of the 1952 ACM national meeting (Toronto) on \ {}- ACM
$\text{\textgreek{’}}$52, 1952, 1–5. https://doi.org/10.1145/ 800259.808980. 

Scheifler, Robert W., and Jim Gettys. “The X Window System.” ACM Transactions on Graphics 5, no. 2 (1986): 79–109.
https://doi.org/10.1145/229 49.24053. 

SCO GROUP, INC. v. INTERN. BUSINESS MACHINES CORP. (https:// www.leagle.com/decision/infco20180102045 January 2, 2018). 

“Share (Computing).” Wikipedia, May 12, 2023. https://en.wikipedia.org/ wiki/SHARE\_(computing). 

“Share Operating System.” Wikipedia, August 13, 2023. https://en.wikipe dia.org/wiki/ SHARE\_Operating\_System. 

Stallman, Richard. “Initial Announcement - GNU Project - Free Software Foundation.” Initial Announcement, September 27,
1983. https://www.gnu.org/ gnu/initial-announcement.html. 

Stallman, Richard. “The GNU Manifesto - GNU Project - Free Software Foundation.” The GNU Manifesto, March 1985.
https://www.gnu.org/gnu/ manifesto.en.html. 

Stallman, Richard. “Why Open Source Misses the Point of Free Software - GNU Project - Free Software Foundation.” Why
Open Source Misses the Point of Free Software, 2023.
https://www.gnu.org/philosophy/open-source-misses-the-point.en.html. 

“The DECUS Tapes.” Index of /pub/academic/computer-science/history/ PDP-11/RSX/decus. Accessed November 22, 2023.
http://www.ibiblio.org/ pub/academic/computer-science/history/pdp-11/rsx/decus/. 

THE SCO GROUP, INC. v. NOVELL, INC. SCO Grp v. Novell Inc (https://www.leagle.com/decision/infdco20110204657 June 10,
2010). 

“Usenet.” Wikipedia, November 18, 2023. https://en.wikipedia.org/wiki/ Usenet. 

van Rossum, Guido. “Open Source Summit Trip Report.” Open Source Summit LG \#28, May 1998.
https://linuxgazette.net/issue28/rossum.html. 

Weber, Steven. Essay. In The Success of Open Source, 38–44. Cambridge, MA: Harvard University Press, 2005. 

“What Is Copyleft? - GNU Project - Free Software Foundation.” What is Copyleft?, 2022.
https://www.gnu.org/licenses/copyleft.html. 


\bigskip


\bigskip
\end{document}
