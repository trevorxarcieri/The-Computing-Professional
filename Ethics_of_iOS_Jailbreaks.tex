% This file was converted to LaTeX by Writer2LaTeX ver. 1.9.9
% see http://writer2latex.sourceforge.net for more info
\documentclass{article}
\usepackage{calc,amsmath,amssymb,amsfonts}
\usepackage[LGR,T1]{fontenc}
\usepackage[greek,english]{babel}
\usepackage[style=numeric,backend=biber]{biblatex}
\date{2023-12-19}
\begin{document}
\clearpage

\bigskip


\bigskip


\bigskip


\bigskip


\bigskip


\bigskip


\bigskip

THE ETHICS OF IOS JAILBREAKS


\bigskip


\bigskip


\bigskip


\bigskip


\bigskip


\bigskip


\bigskip

Trevor Arcieri

Computer Science 300: The Computing Professional

December 16, 2023
\clearpage

\clearpage

\ \ Like many in my generation, I got my first mobile device in elementary school. It was a 4th generation iPod touch
that my dad found on Craigslist. In the item description, he saw that the iPod was
{\textquotedbl}jailbroken{\textquotedbl} but had no clue what that meant or what it could mean for my future. This
particular device had an untethered jailbreak, which means the jailbreak persisted through a device reboot.

\ \ The process is more complicated than this, as the jailbreak included a modification to the device's boot loader,
which would enable root access before booting up the operating system. In other words, a change in the code run when
the device turns on would re-hack into the operating system. Each time the device turned on, I would watch in awe as an
abundance of green terminal text would scroll across the screen.

\ \ Now, as a Computer Science and Computer Engineering dual major with an Electrical Engineering minor at Kettering
University in the 3rd year of my studies, and one with a profuse interest in all things computing at that, I look back
on those moments watching that green text as the first times I had an interest in the inner-workings of the
seemingly-simple technology all around me. Seeing the text, which was too small and scrolling too fast to read any
significant amount, I would ask myself, {\textquotedbl}What's going on in there?{\textquotedbl} As I will detail
shortly, this experience of gaining an interest in computing from an iOS jailbreak is a direct result of the nature of
the jailbreak.

\ \ To be clear, what is iOS jailbreaking? At its core, it uses a privilege escalation exploit to remove restrictions
within iOS, Apple's mobile operating system; the exploit can remove these restrictions because it allows root access to
the operating system.\footnote{“IOS Jailbreaking,” Wikipedia, December 16, 2023,
https://en.wikipedia.org/wiki/IOS\_jailbreaking.} An exploit is software that uses an existing vulnerability to cause
unintended behavior in a computer system.\footnote{“Exploit,” TREND MICRO, 2023,
https://www.trendmicro.com/vinfo/us/security/definition/exploit.} Root access, then, is access to the operating system
as the administrator, allowing the user to make unrestricted changes to the system.\footnote{“Superuser,” Wikipedia,
December 15, 2023, https://en.wikipedia.org/wiki/Superuser.}

\ \ Jailbreak exploits, in particular, are privilege escalation exploits that gain elevated access to protected
resources in the operating system.\footnote{“Privilege Escalation,” Wikipedia, November 17, 2023,
https://en.wikipedia.org/wiki/Privilege\_escalation.} Jailbreak exploits usually work through kernel patches, which are
modifications to the kernel. Modifying the kernel creates an extremely powerful exploit since the kernel sits at the
core of the operating system and controls everything in the system.\footnote{“Kernel Definition,” The Linux Information
Project, May 31, 2005, https://www.linfo.org/kernel.html.}

\ \ What, then, does jailbreaking do? The main result of jailbreaking an iOS device is access to free software
installation. In this case, free software installation does not mean free in the monetary sense because some software
installed after a jailbreak can cost money. Instead, it is free software installation in the Richard Stallman of GNU
sense: entirely unrestricted.\footnote{“IOS Jailbreaking.”} Primarily, a jailbreak allows the installation of apps
without restrictions from App Store guidelines.\footnote{“Agreements and Guidelines for Apple Developers,” Apple
Developer, 2023, https://developer.apple.com/support/terms/; Associated Press, “Jailbreak! New Rules Allow Unapproved
Iphone Apps,” Fox News, October 22, 2015,
https://www.foxnews.com/tech/jailbreak-new-rules-allow-unapproved-iphone-apps; Chris Forseman, “IPhone Jailbreaker Set
to Bring Cydia to Mac OS X,” Ars Technica, December 14, 2010,
https://arstechnica.com/gadgets/2010/12/iphone-jailbreaker-set-to-bring-cydia-to-mac-os-x/.} Since a device with root
access installs these apps, the apps can manipulate restricted areas of the operating system, such as the user
interface.\footnote{“IOS Jailbreaking.”}

\ \ The perks of jailbreaking, though, do not come without risks$\text{\textgreek{—}}$the main trouble lies in the
definition of a jailbreak: gaining root access to iOS. Gaining root access to any operating system is extremely
dangerous, and even many standard Linux programs explicitly warn against running with root access. As previously noted,
root access allows user programs to manipulate the entire OS as a superuser, enabling access to all files and programs
on the device. Some of these files and programs are restricted from user access out of necessity by Apple developers
since modification could damage the device.

\ \ Thus, jailbreaking is dangerous because the installed jailbreak software has access to the entire operating system,
allowing the unregulated jailbreak software to perform any action within the system. Further, jailbreaking is
exponentially more dangerous due to the unrestricted software installation that comes along with a jailbreak, as any of
the unregulated programs installed onto the device also have root access to the operating system.

\ \ Jailbreaking was popular in the mass media, especially in its early era, so many parties have cultivated an opinion
on the topic. Apple, for one, is staunchly opposed to jailbreaking, at least publicly. In the iPhone user guide section
covering unauthorized modification of iOS, Apple expresses this opposition. They claim that jailbreaking is harmful
because it eliminates security layers within the operating system; can cause increased battery drain and system
crashes; can disrupt services like iCloud, iMessage, and FaceTime; can preclude the device from software updates since
an update could make a device permanently inoperable (colloquially known as {\textquotedbl}bricking{\textquotedbl} the
machine); can cause irreversible damage to iOS; and violates the iOS software license agreement and thus the Apple End
User License Agreement (EULA).\footnote{“Unauthorized Modification of IOS,” Apple Support, 2023,
https://support.apple.com/guide/iphone/unauthorized-modification-of-ios-iph9385bb26a/ios.}

\ \ On the other hand, jailbreak users, and thus some of Apple's users, vehemently favor jailbreaking. The unrestricted
software packages enabled for installation following a jailbreak are commonly called
{\textquotedbl}tweaks.{\textquotedbl} They are managed by some package-management software, with the most common
package manager being Cydia; the package manager serves as a sort of {\textquotedbl}app store{\textquotedbl} for
tweaks.\footnote{Adam Dachis, “How to Get the Most out of Your Jailbroken IOS Device,” Lifehacker, March 14, 2011,
https://web.archive.org/web/20151225202409/http://lifehacker.com/5781437/how-to-get-the-most-out-of-your-jailbroken-ios-device.}

\ \ Some tweaks serve as alternative or better versions of apps in the App Store; others allow configuration of the
operating system in many ways, namely in specific and beneficial user interface features.\footnote{Dachis, “Your
Jailbroken IOS Device.”} Both forms of tweaks are desirable to the user. Further, Apple censors certain apps in the App
Store,\footnote{“Rejected Apps,” iMore, 2014,
https://web.archive.org/web/20140715085458/http://www.imore.com/tag/rejected-apps.} which frustrates many of its
users.\footnote{Steve Kovach, “Frustration Builds with Apple$\text{\textgreek{’}}$s Inconsistent Rules for App
Developers,” Business Insider, April 13, 2013,
https://www.businessinsider.com/the-story-of-apples-confusing-inconsistent-rules-for-app-developers-2013-4.} On Cydia,
there is no censorship of tweaks. Furthermore, over 60 of the most memorable features implemented on iOS since its 5.0
release were features that the jailbreaking community first released on Cydia: features we use every day, from the more
noticeable like emoji support, dark mode, predictive text, and control center, to the less prominent like interactive
notifications, a button to clear all notifications at once, and the one-handed keyboard.\footnote{Amboy Manalo, “60 IOS
Features Apple Stole from Jailbreakers,” Gadget Hacks, November 2, 2018,
https://ios.gadgethacks.com/how-to/60-ios-features-apple-stole-from-jailbreakers-0188093/; Oliver Haslam, “9 Jailbreak
Tweaks Apple Killed in IOS 9,” Redmond Pie, June 10, 2015,
https://www.redmondpie.com/9-jailbreak-tweaks-apple-killed-in-ios-9/.} With the jailbreaking community first developing
all these features, it follows that users would enjoy access to the new and advanced features that jailbreaking
permits.

\ \ With strong opinions on each side, it is essential to ask: Is it ethical to jailbreak your iOS device? To answer
this question, it is first necessary to consider the rights and obligations of each party involved. The rights and
obligations of Apple regarding jailbreaking must be considered.

\ \ First, Apple owns its proprietary iOS operating system and can restrict its usage via the iOS software license
agreement. Thus, Apple has the right to limit the functionality of iOS. Similarly, since Apple is selling products
containing their licensed software, Apple has the right to enforce that license regarding the use of the software on
their devices. In terms of obligations, Apple's primary goal and duty as a corporation is to turn a profit for its
shareholders. This obligation to turn a profit leads to other commitments as well. To keep their users (and thus their
profit), though, Apple is obligated to some extent to provide innovative features and customization within their
software. Finally, to maintain its public reputation (and its profit), Apple must produce safe software for its users.


\bigskip

\ \ All Apple users' rights and obligations regarding jailbreaking must be considered. Importantly, this party includes
all of Apple$\text{\textgreek{’}}$s users, only a subset of which, jailbreak users, actually jailbreak their devices.
First, every Apple user has the right to customize their device to an extent: the user owns the physical device, and it
is only human nature to customize that which one holds. Additionally, since each user is paying a premium for an Apple
device, it is reasonable to conclude that the user has the right to new and valuable features on iOS. Such features
improve the user$\text{\textgreek{’}}$s quality of life in their interactions with the device, and a premium device
necessitates a premium quality of life. The primary obligation the user holds to Apple is to uphold the EULA that they
agree to abide by regarding the usage of their Apple device.

\ \ In response to jailbreaking your Apple device, there are three main options. The first is not jailbreaking your
device whatsoever. The second is jailbreaking your device but only installing minimal tweaks. The third is jailbreaking
and using your device without restriction, installing whatever tweaks you desire.

\ \ With these three options considered, it is helpful to analyze each alternative concerning the rights and obligations
of each party involved. First, the no-jailbreak option must be explored. This option respects Apple's right to make iOS
restrictions and to enforce their software license. Further, it fulfills Apple's obligation to make a profit, at least
in the short term, since it ensures users only install the apps they intend, thus bolstering any profit Apple receives
from approving these apps. It also wholly upholds Apple's obligation to provide safe software since Apple has complete
control over the software on the device in this case. Regarding Apple's commitment to providing innovative features,
though, this option needs to be improved: the only features available to users are those developed by Apple's
developers, which restricts the total set of feature options to users.

\ \ In regards to the user, not jailbreaking your device has separate impacts. For one, it limits the user's right to
customize their device. Likewise, it violates the user's right to new and valuable features. Still, this option does
fulfill the user's obligation to follow the EULA since the user is not using the device in any unsanctioned manner.

\ \ Second, the jailbreak-with-minimal-tweaks option must be analyzed. While it does violate Apple's right to make
restrictions on the functionality of iOS in modifying the operating system with the jailbreak, since the user installs
only minimal tweaks on the device, the violation of Apple's restrictions is minimal. Apple maintains the right to
enforce its software license should it discover the device is jailbroken. While the jailbreak may diminish Apple's
short-term profits by providing the user with an alternative method of app installation, it also may increase Apple's
profit in the long term as the user is contributing to a jailbreak community (simply by using tweaks or possibly even
developing them) which produces new features which Apple may implement in the future. This option also helps fulfill
Apple's obligation to provide innovative features to the user since the user may install a small number of beneficial
tweaks without detracting from Apple's development power (as the jailbreak community develops the tweaks rather than
Apple's developers). Finally, Apple's obligation to provide safe software is at risk since the jailbreak may remove
security and safeguard features within the operating system. Still, with only minimal tweaks, the exponential impact of
further risks due to additional unrestricted software installation is minimal.

\ \ It is likewise essential to analyze the rights and obligations of the user regarding jailbreaking your device with
minimal tweaks. This option respects the user's right to customize their device but does not fulfill the request to the
greatest extent since the user can only install minimal tweaks. Similarly, this option partially fulfills the user's
right to new features since the user may install a few ground-breaking tweaks. Finally, this option does violate the
user's obligation to Apple to uphold the EULA since jailbreaking their device violates this agreement. However, only
installing minimal tweaks keeps the violation to a minimal magnitude.

\ \ Third, the unrestricted jailbreak option must be analyzed. This option violates Apple's right to make iOS
restrictions to the fullest extent since users may circumvent any limits they desire. Again, this option does not
infringe on Apple's right to enforce its software license since Apple can still enforce the agreement on the user
should they discover the device is jailbroken. This option also diminishes Apple's short-term ability to make a profit
to the greatest extent since the user is free to install whatever software they wish, which will likely detract from
the amount of Apple-approved software the user installs. Still, since the user is installing more tweaks, this option
contributes most to Apple's long-term ability to make a profit since the user contributes highly to the jailbreak
community's innovative features, which Apple may later utilize. Further, this option fulfills most of Apple's
obligation to provide innovative features since the user is free to install unlimited tweaks, many of which are on the
cutting edge of user experience and all of which are sourced from the entirety of the jailbreak community, rather than
any of Apple's development bandwidth.

\ \ The rights and obligations of the user are crucial to analyze as well. Notably, this option provides the user the
most significant opportunity to exercise their right to customize their device since the user can install whatever
customizations they desire. Similarly, this option is the best in fulfilling the user's right to enjoy new features on
iOS since they can enjoy the fullest extent of new features offered by the jailbreak community. Finally, this option
violates the user's obligation not to break the EULA and does so reasonably strongly since the user may install
whatever software they wish. Still, it is essential to note that since jailbreak users are a subset of all Apple's
users, this option does not violate Apple's EULA en masse; instead, it only breaks it at a local level.


\bigskip

\ \ Now, in evaluating these options, it is helpful to consider the application of Social Contract Theory. In this
instance, the application assumes a representative of Apple and a representative of Apple's users enter a room to
discuss the issue. Still, upon entering, they forget which party they represent. Thus, the outcome of the discussion
will protect the most vulnerable party involved since either representative could be a member of that weak party.

\ \ In such a discussion, both individuals would identify that the user, fully susceptible to what Apple permits in its
software, is the most vulnerable. Further, both parties would find Apple's profit and reputation robust to users
jailbreaking their devices since jailbreak users are a mere subset of Apple's total users. Thus, faced with the three
options mentioned earlier, the parties would agree on the free jailbreak: it generates the most significant profit for
Apple in the long term while wholly protecting the user's rights to enjoy their device uniquely.

\ \ Other ethical frameworks help judge these options as well. Namely, the doctrines of utilitarianism (focusing on the
outcome), deontological theory (focusing on the actions), and virtue ethics (focusing on the growth of the parties
involved) are essential.

\ \ First, not jailbreaking your phone can be considered within these frameworks. Utilitarianism would judge this option
as ethical since this is the only option that does not violate the EULA because it is the only option that does not put
jailbreak software on the device. From a deontological theory perspective, this option is ethical since the user uses
their phone as usual. Using their phone as usual does not raise any ethical concerns, as this is what Apple intended
when selling the device to the user. Finally, in terms of virtue ethics, this option is highly unethical. Apple's
potential is limited since the user does not have the chance to support the jailbreak community's features, and,
likewise, the user's potential is limited because they are not able to gain the computing skills, UI design skills, and
possibly even the tweak development skills they would have achieved with a jailbreak.

\ \ Next, jailbreaking your phone with minimal tweaks may be considered via these ethical frameworks. To address
utilitarianism, the outcome of the user violating the EULA is unethical. Still, due to the impact of a small number of
jailbreaks among their users being somewhat limited, this option is not egregiously unethical. Further, from a
utilitarian standpoint, there is even a slight ethical upside since the minimal jailbreak allows the user to indulge in
some tweaks to their system. Deontological theory also would judge this option as unethical, albeit to a small extent.
Regardless of the number of tweaks, the user is taking actions prohibited by Apple, making the option unethical. Still,
the user's actions, especially with these minimal tweaks, are working to balance their needs for features with Apple's
need for a safe operating system as best as possible. From a virtue ethics framework, this option is ethical but not
the most favorable. It allows Apple to dabble in some benefits from jailbreak features while the user benefits from
some technological understanding, as was missed in the no-jailbreak option.

\ \ Third, jailbreaking your phone and using it without restrictions can be judged in these frameworks. From the
perspective of utilitarianism, the EULA is violated, as in the minimal jailbreak option, which is unethical to an
extent. Yet, the jailbreak results in Apple and the user enjoying the benefits of the features, which provides
significant benefits to both parties, so utilitarianism reservedly supports this option. Regarding deontological
theory, the actions are unethical because Apple prohibits them. However, in jailbreaking their device, the user
considers their needs for features to the fullest extent, which means this framework also conservatively supports the
option. From the virtue ethics standpoint, this option is very favorable. With this option, Apple gets the most benefit
in terms of features due to the jailbreak user becoming entrenched in their iOS knowledge as they dive further into
various tweaks. Similarly, the user gets the highest benefit of technological acumen with this option, as the user can
utilize the jailbreak to the fullest extent possible, learning along the way.

\ \ \ \ \ \ Ultimately, the most ethical option is an unrestricted jailbreak, with a caveat: the user must actively
learn from and participate in the jailbreak community. The user jailbreaking their device does violate EULA, which goes
against deontological theory. Yet, the user gets a highly-customizable and thus the best experience, which satisfies
utilitarianism. Further, Apple can learn from new features used and developed by tech-savvy jailbreak users, which
satisfies virtue ethics. Additionally, the social contract analysis confirms the unrestricted jailbreak as the best
option because the negative impact is negligible on Apple if tech-savvy individuals use the jailbreak and refrain from
publicizing issues caused by it. The positive effects on users and Apple from enjoying the spoils of jailbreak features
are rather 

significant.

\ \ Regarding downloading specific tweaks being ethical, the question would have to be case by case. Some tweaks are UI
customizations, while others allow pirate versions of paid App Store apps and free in-app purchases. Regardless,
allowing a tech-savvy user unrestricted access to these tweaks is ethical. In all reality, the tech-savvy are not using
jailbreaks for pirating but more for iOS development and personal UI customization.\footnote{Jonathan Zdziarski, IPhone
Open Application Development: Write Native Applications Using the Open Source Tool Chain (Sebastopol:
O$\text{\textgreek{’}}$Reilly Media, Inc., 2008); Ted Landau, Take Control of Your Iphone (TidBITS Publishing, Inc,
2009).}

\ \ Ultimately, jailbreaking your iPhone helps you with technological insight and allows Apple to work even more
intensely as tweaks are tested and further developed. Thus, it is only logical that jailbreaking your iPhone is
ethical, as it forwards humanity's technological progression.


\bigskip

\clearpage
Bibliography

“Agreements and Guidelines for Apple Developers.” Apple Developer, 2023. https://developer.apple.com/support/terms/.

Associated Press. “Jailbreak! New Rules Allow Unapproved Iphone Apps.” Fox News, October 22, 2015.
https://www.foxnews.com/tech/jailbreak-new-rules-allow-unapproved-iphone-apps.

Dachis, Adam. “How to Get the Most out of Your Jailbroken IOS Device.” Lifehacker, March 14, 2011.
https://web.archive.org/web/20151225202409/http ://lifehacker.com/
5781437/how-to-get-the-most-out-of-your-jailbroken-ios-devi ce.

“Exploit.” TREND MICRO, 2023. https://www.trendmicro.com/vinfo/us/ security/definition/exploit.

Forseman, Chris. “IPhone Jailbreaker Set to Bring Cydia to Mac OS X.” Ars Technica, December 14, 2010.
https://arstechnica.com/gadgets/2010/12/ iphone-jailbreaker-set-to-bring-cydia-to-mac-os-x/.

Haslam, Oliver. “9 Jailbreak Tweaks Apple Killed in IOS 9.” Redmond Pie, June 10, 2015.
https://www.redmondpie.com/9-jailbreak-tweaks-apple-killed-in-ios-9/.

“IOS Jailbreaking.” Wikipedia, December 16, 2023. https://en.wikipedia. org/wiki/IOS\_jailbreaking.

“Kernel Definition.” The Linux Information Project, May 31, 2005. https:// www.linfo.org/kernel.html.

Kovach, Steve. “Frustration Builds with Apple$\text{\textgreek{’}}$s Inconsistent Rules for App Developers.” Business
Insider, April 13, 2013. https://www.businessinsider.com
/the-story-of-apples-confusing-inconsistent-rules-for-app-developers-2013-4.

Landau, Ted. Take control of your iphone. TidBITS Publishing, Inc, 2009.

Manalo, Amboy. “60 IOS Features Apple Stole from Jailbreakers.” Gadget Hacks, November 2, 2018.
https://ios.gadgethacks.com/how-to/60-ios-features-apple-stole-from-jailbreakers-0188093/.

“Privilege Escalation.” Wikipedia, November 17, 2023. https://en.wikipedia .org/wiki/ Privilege\_escalation.

“Rejected Apps.” iMore, 2014. https://web.archive.org/web/201407150854 58/http://www.imore.com/tag/rejected-apps.

“Superuser.” Wikipedia, December 15, 2023. https://en.wikipedia.org/wiki /Superuser.

“Unauthorized Modification of IOS.” Apple Support, 2023. https://support. apple.com/guide/iphone/
\ unauthorized-modification-of-ios-iph9385bb26a/ios.

Zdziarski, Jonathan. IPhone open application development: Write native applications using the Open Source Tool Chain.
Sebastopol: O$\text{\textgreek{’}}$Reilly Media, Inc., 2008.
\end{document}
